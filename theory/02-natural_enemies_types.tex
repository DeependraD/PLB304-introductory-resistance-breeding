\PassOptionsToPackage{unicode=true}{hyperref} % options for packages loaded elsewhere
\PassOptionsToPackage{hyphens}{url}
\documentclass[10pt,dvipsnames,ignorenonframetext,aspectratio=169]{beamer}
\IfFileExists{pgfpages.sty}{\usepackage{pgfpages}}{}
\setbeamertemplate{caption}[numbered]
\setbeamertemplate{caption label separator}{: }
\setbeamercolor{caption name}{fg=normal text.fg}
\beamertemplatenavigationsymbolsempty
\usepackage{lmodern}
\usepackage{amssymb,amsmath}
\usepackage{ifxetex,ifluatex}
\usepackage{fixltx2e} % provides \textsubscript
\ifnum 0\ifxetex 1\fi\ifluatex 1\fi=0 % if pdftex
  \usepackage[T1]{fontenc}
  \usepackage[utf8]{inputenc}
\else % if luatex or xelatex
  \ifxetex
    \usepackage{mathspec}
  \else
    \usepackage{fontspec}
\fi
\defaultfontfeatures{Ligatures=TeX,Scale=MatchLowercase}







\fi

  \usetheme[]{monash}

  \usecolortheme{monashwhite}


% A default size of 24 is set in beamerthememonash.sty
  \setbeamerfont{title}{series=\bfseries,parent=structure,size=\fontsize{18pt}{32}}

% Title page
\setbeamertemplate{title page}
{\placefig{-0.01}{-0.01}{width=1.01\paperwidth,height=1.01\paperheight}{pentatomid\_andrallus\_bug\_20211023.jpg}
\begin{textblock}{7.5}(1,2.8)\usebeamerfont{title}
{\color{white}\raggedright\par\inserttitle}
\end{textblock}
\begin{textblock}{7.5}(1,7)
{\color{white}\raggedright{\insertauthor}\mbox{}\\[0.2cm]
\insertdate}
\end{textblock}}


  \useinnertheme{rounded}

  \useoutertheme{smoothtree}

% use upquote if available, for straight quotes in verbatim environments
\IfFileExists{upquote.sty}{\usepackage{upquote}}{}
% use microtype if available
\IfFileExists{microtype.sty}{%
  \usepackage{microtype}
  \UseMicrotypeSet[protrusion]{basicmath} % disable protrusion for tt fonts
}{}


\newif\ifbibliography
  \usepackage[round]{natbib}
  \bibliographystyle{plainnat}


\hypersetup{
      pdftitle={Natural Enemies and their Types},
            colorlinks=true,
    linkcolor=red,
    citecolor=Blue,
    urlcolor=lightgrayd,
    breaklinks=true}
%\urlstyle{same}  % Use monospace font for urls







% Prevent slide breaks in the middle of a paragraph:
\widowpenalties 1 10000
\raggedbottom

  \AtBeginPart{
    \let\insertpartnumber\relax
    \let\partname\relax
    \frame{\partpage}
  }
  \AtBeginSection{
    \ifbibliography
    \else
      \let\insertsectionnumber\relax
      \let\sectionname\relax
      \frame{\sectionpage}
    \fi
  }
  \AtBeginSubsection{
    \let\insertsubsectionnumber\relax
    \let\subsectionname\relax
    \frame{\subsectionpage}
  }



\setlength{\parindent}{0pt}
\setlength{\parskip}{6pt plus 2pt minus 1pt}
\setlength{\emergencystretch}{3em}  % prevent overfull lines
\providecommand{\tightlist}{%
  \setlength{\itemsep}{0pt}\setlength{\parskip}{0pt}}

  \setcounter{secnumdepth}{0}


%% Monash overrides
\AtBeginSection[]{
   \frame<beamer>{
   \frametitle{Outline}\vspace*{0.2cm}
   
   \tableofcontents[currentsection,hideallsubsections]
  }}

% Redefine shaded environment if it exists (to ensure text is black)
\ifcsname Shaded\endcsname
  \definecolor{shadecolor}{RGB}{225,225,225}
  \renewenvironment{Shaded}{\color{black}\begin{snugshade}\color{black}}{\end{snugshade}}
\fi
%%


  \usepackage{setspace}
  \usepackage{wasysym}
  % \usepackage{footnote} % don't use this this breaks all
  \usepackage{fontenc}
  \usepackage{fontawesome}
  \usepackage{booktabs,siunitx}
  \usepackage{longtable}
  \usepackage{array}
  \usepackage{multirow}
  \usepackage{wrapfig}
  \usepackage{float}
  \usepackage{colortbl}
  \usepackage{pdflscape}
  \usepackage{tabu}
  \usepackage{threeparttable}
  \usepackage{threeparttablex}
  \usepackage[normalem]{ulem}
  \usepackage{makecell}
  \usepackage{xcolor}
  \usepackage{tikz} % required for image opacity change
  \usepackage[absolute,overlay]{textpos} % for text formatting
  \usepackage{chemfig}
  \usepackage[skip=0.333\baselineskip]{caption}
  % \newcommand*{\AlignChar}[1]{\makebox[1ex][c]{\ensuremath{\scriptstyle#1}}}%
  \usepackage{siunitx}

  % this font option is amenable for beamer
  \setbeamerfont{caption}{size=\tiny}
  \singlespacing
  \definecolor{lightgrayd}{gray}{0.95}
  \definecolor{skyblued}{rgb}{0.65, 0.6, 0.94}
  \definecolor{oranged}{RGB}{245, 145, 200}

  % % better to insert it into template itself
  % \newlength{\cslhangindent}
  % \setlength{\cslhangindent}{1.5em}
  % \newenvironment{cslreferences}%
  %   {\setlength{\parindent}{0pt}%
  %   \everypar{\setlength{\hangindent}{\cslhangindent}}\ignorespaces}%
  %   {\par}

  \usepackage[caption=false]{subfig}

  \newcommand{\bcolumns}{\begin{columns}[T, onlytextwidth]}
  \newcommand{\ecolumns}{\end{columns}}

  \newcommand{\bdescription}{\begin{description}}
  \newcommand{\edescription}{\end{description}}

  \newcommand{\bitemize}{\begin{itemize}}
  \newcommand{\eitemize}{\end{itemize}}
  \AtBeginSubsection{}

  \title[]{Natural Enemies and their Types}


  \author[
        \vspace{-1cm}Deependra Dhakal\\
Assistant Professor\\
Agriculture and Forestry University\\
\textit{ddhakal.rookie@gmail.com}\\
\url{https://rookie.rbind.io}
    ]{\vspace{-1cm}Deependra Dhakal\\
Assistant Professor\\
Agriculture and Forestry University\\
\textit{ddhakal.rookie@gmail.com}\\
\url{https://rookie.rbind.io}}


\date[
      
  ]{
    }

\begin{document}

% Hide progress bar and footline on titlepage
  \begin{frame}[plain]
  \titlepage
  \end{frame}


   \frame<beamer>{
   \frametitle{Outline}\vspace*{0.2cm}
   
   \tableofcontents[hideallsubsections]
  }

\hypertarget{natural-enemy-classification}{%
\section{Natural Enemy:
Classification}\label{natural-enemy-classification}}

\begin{frame}{}
\protect\hypertarget{section}{}
\begingroup\fontsize{8}{10}\selectfont

\begin{longtable}[t]{lll}
\caption{\label{tab:natural-enemy-classification}Taxonomic group of organisms and nature of damage to crops}\\
\toprule
Taxonomic group & Enemy group & Effect on plant\\
\midrule
\cellcolor{gray!6}{Virus} & \cellcolor{gray!6}{Pathogen} & \cellcolor{gray!6}{Disease infection}\\
Phytoplasma & Pathogen & Disease infection\\
\cellcolor{gray!6}{Fungus} & \cellcolor{gray!6}{Pathogen} & \cellcolor{gray!6}{Disease infection}\\
Higher plant & Parasite & Infestation\\
\cellcolor{gray!6}{Nematode} & \cellcolor{gray!6}{Parasite} & \cellcolor{gray!6}{Infestation}\\
\addlinespace
Insect & Parasite & Infestation\\
\cellcolor{gray!6}{Insect} & \cellcolor{gray!6}{Herbivores} & \cellcolor{gray!6}{Infestation}\\
Snail and slug & Herbivores & Infestation\\
\cellcolor{gray!6}{Vertebrate} & \cellcolor{gray!6}{Herbivores} & \cellcolor{gray!6}{Biting damage}\\
\bottomrule
\end{longtable}
\endgroup{}
\end{frame}

\begin{frame}{Steps in classical biological control
\citep{van2016integrating}}
\protect\hypertarget{steps-in-classical-biological-control-van2016integrating}{}
\begin{enumerate}
\tightlist
\item
  Target selection: Ecological rationale for engaging in project;
  obtaining financing
\item
  Species confirmations: Verify identity of pest and, later, its
  potential natural enemies
\item
  Survey for natural enemies: Carried out initially in the invaded area
\item
  Identifying pest's native range: To locate areas to search for natural
  enemies
\item
  Collecting natural enemies in native range: To obtain candidate
  natural enemies
\item
  Judging potential efficacy and host range: Field estimates to guide
  initial selection of natural enemies
\item
  Establishing quarantine colonies: To preserve stocks of natural
  enemies of likely value
\item
  Estimating host ranges: Done for natural enemies of interest, usually
  in quarantine laboratories in receiving countries
\item
  Petitioning for release: Done for natural enemies believed safe and
  potentially effective
\item
  Release and establishment: Carried out in various parts of the invaded
  range
\item
  Post-release monitoring: To determine impacts on pest and non-target
  species
\item
  Assessing program outcomes: Assessment of the completeness and value
  of results via monitoring of pest and recovery of the invaded
  community.
\end{enumerate}
\end{frame}

\begin{frame}{Insects and Diseases of Major Crops in Nepal}
\protect\hypertarget{insects-and-diseases-of-major-crops-in-nepal}{}
\begin{table}[H]

\caption{\label{tab:insect-pests-major}Major disease and insect pests of major cultivated crops in Nepal}
\centering
\fontsize{6}{8}\selectfont
\begin{tabular}[t]{l>{\raggedright\arraybackslash}p{20em}>{\raggedright\arraybackslash}p{20em}}
\toprule
Crop & Major insects & Major disease\\
\midrule
\cellcolor{gray!6}{Rice} & \cellcolor{gray!6}{Rice bug, rice hispa, yellow stem borer, stripped stem borer, rice gall midge, mole cricket, plant hopper} & \cellcolor{gray!6}{Bacterial blight (Xanthomonas oryzae), Blast (Pyricularia oryzae), False smut (Ustilaginoides virens), Brown leaf spot (Helminthosporium oryzae)}\\
Wheat & Armyworm, cutworm, shoot fly, stem borer, termites & Leaf spots (Helminthosporium spp), rust, leaf streak (Xanthomonas spp), loose smut\\
\cellcolor{gray!6}{Maize} & \cellcolor{gray!6}{Stalk borer, shoot fly, cutworm, jassid, armyworm} & \cellcolor{gray!6}{Rust, leaf blight (Helminthosporium maydis), smut (Specealothica reliana)}\\
Barley & Green bug, corn sawfly, fruitfly, wheat bulb fly & Barley yellow dwarf virus, powdery mildew (Erysiphe graminis sp. hordii), Net blotch (Helminthosporium sativum)\\
\bottomrule
\end{tabular}
\end{table}
\end{frame}

\hypertarget{diseases}{%
\section{Diseases}\label{diseases}}

\begin{frame}{Pathogenic fungi}
\protect\hypertarget{pathogenic-fungi}{}
\begin{itemize}
\tightlist
\item
  Most belong to phyla Ascomycota and Basidiomycota
\item
  Fungi reproduce both sexually and asexually via the production of
  spores and other structures
\item
  Basidiomycetes

  \begin{itemize}
  \tightlist
  \item
    produce sexual spores called basidiospores, on a club-shaped
    spore-producing structure called a basidium
  \item
    mostly are fleshy fungi such as common mushrooms, the puffballs, and
    the shelf fungi or conks, and are either saprophytes or cause wood
    decay, including root and stem rots of trees.
  \item
    two most destructive groups of plant pathogenic fungi -- rust
    (Pucciniomycotina) and smut (Ustilaginomycotina) -- are contained
  \item
    characterized by basidia as meiosporocysts in the sexual life stage
  \item
    karyogamy and meiosis proceed in the basidia and basidiospores are
    produced
  \item
    basidiomycetous hyphae, which have an electron-dense (multi-layered
    or visually single-layered) wall, are divided by septa into
    mononucleate, binucleate, or multinucleate segments
  \item
    according to Ainsworth and Bisby's Dictionary of Fungi (2008), there
    are 1589 genera and more than 30,000 species of Basidiomycoa (32\%
    of all described fungal taxa)
  \end{itemize}
\end{itemize}
\end{frame}

\begin{frame}{Rust and smut}
\protect\hypertarget{rust-and-smut}{}
\begin{itemize}
\tightlist
\item
  Rust fungi are obligate parasites in nature
\item
  Notorious on grain crops (wheat, oats, barley), vegetables (bean,
  asparagus), field crops (cotton, soybeans), ornamentals (carnation,
  chrysanthemum, snapdragon) and trees (pine, apple, coffee)
\item
  Infection usually appear numerous rusty, orange, yellow, or even
  white-colored spots that rupture the epidermis. Some form swellings
  and even galls. Most rust infections are strictly local spots, but
  some may become systemic.
\item
  There are about 5000 species of rust fungi.
\item
  Sexual reproduction tends to promote rapid development of new races.
\item
  Rust caused by fungi that produce only teliospores and basidiospores
  are called microcyclic or short cycled.
\end{itemize}
\end{frame}

\begin{frame}{Common rust fungi in agriculture}
\protect\hypertarget{common-rust-fungi-in-agriculture}{}
\begin{itemize}
\tightlist
\item
  \emph{Hemileia vastatrix} (Coffee rust); Primary host is coffee plant;
  unknown alternate host. Heteroecious
\item
  \emph{Phakopsora meibomiae} and \emph{P. pachyrhizi} (Soybean rust);
  Primary host is soybean and various legumes. Unknown alternate host.
  Heteroecious
\item
  \emph{Puccinia coronata} (Crown Rust of Oats and Ryegrass); Oats are
  the primary host; Rhamnus spp. (Buckthorn) is alternate host.
  Heteroecious and macrocyclic
\item
  \emph{Puccinia graminis} (Stem rust of wheat and Kentucky bluegrass,
  or black rust of cereals); Primary hosts include: Kentucky bluegrass,
  barley, and wheat; Common barberry is the alternate host. Heteroecious
  and macrocyclic
\item
  \emph{Puccinia hemerocallidis} (Daylily rust); Daylily is primary
  host; Patrina sp is alternate host. Heteroecious and macrocyclic
\item
  \emph{Puccinia triticina} (Brown Wheat Rust) in grains
\item
  \emph{Puccinia sorghi} (Common Rust of Corn)
\item
  \emph{Puccinia striiformis} (Yellow Rust) of cereals
\item
  \emph{Uromyces appendiculatus} (Bean Rust) in common bean (Phaseolus
  vulgaris)
\item
  \emph{Puccinia melanocephala} (Brown Rust of Sugarcane)
\item
  \emph{Puccinia kuehnii} (Orange rust of Sugarcane)
\end{itemize}
\end{frame}

\begin{frame}{}
\protect\hypertarget{section-1}{}
\begin{itemize}
\tightlist
\item
  There are five spore stages that are produced and two hosts are
  required (heteroecious species) in the completion:

  \begin{itemize}
  \tightlist
  \item
    Stage 0:Spermagonium

    \begin{itemize}
    \tightlist
    \item
      Flask shaped structure containing haploid spermatia (male gamete)
      and receptive hyphae
    \item
      Spermatia fertilize receptive hyphae of the compatible mating type
      and subsequent production of dikaryotic mycelium.
    \item
      Produces (on upper surface of the barberry leaf) sex organs in
      rust
    \item
      Are of two mating types (due to being derived from basidiospores)
    \end{itemize}
  \item
    Stage I: Aecium

    \begin{itemize}
    \tightlist
    \item
      When spermatia are transferred to compatible receptive hyphae,
      this begins the dikaryotic stage of the life cycle and directly
      produces the aecium on the lower surface of the barberry leaf
    \item
      Chains of aeciospores (which burst through the lower surface of
      the leaf) are formed in aecium
    \end{itemize}
  \item
    Stage II: Uredium

    \begin{itemize}
    \tightlist
    \item
      Can attack and infect host plants
    \end{itemize}
  \item
    Stage III: Telium

    \begin{itemize}
    \tightlist
    \item
      Teliospores serve only as the sexual, overwintering stage
    \end{itemize}
  \item
    Stage IV: Basidium

    \begin{itemize}
    \tightlist
    \item
      Heterothallic and basidia produce basidiospores that are of two
      mating types
    \item
      Basidiospores are capable of only infecting the leaves of
      \textit{Berberis} sp. (barberry)
    \item
      Cells of the teliospore germinates to produce a short germ tube
      that will develop into a basidium that is essentially transversely
      septate
    \end{itemize}
  \end{itemize}
\end{itemize}
\end{frame}

\begin{frame}{}
\protect\hypertarget{section-2}{}
\begin{columns}[T, onlytextwidth]
\column{0.35\textwidth}

\begin{figure}
\includegraphics[width=0.9\linewidth]{../images/spores_rust_fungi} \caption{Kinds and sequence of spores and spore-producing structures in rust fungi along with the nuclear condition of each. (Myc: Mycelium)}\label{fig:rust-spores-structures}
\end{figure}

\column{0.65\textwidth}

\begin{figure}
\includegraphics[width=0.85\linewidth]{../images/asexual_multicellular_structures_fungi} \caption{Asexual complex multicellular structures produced by fungi in the Pucciniomycotina (spermogonium, aecium) and Ascomycota (acervulus, pycnidium, sporodochium, coremium = synnemata). Note that these structures, like all complex multicellular structures, have a genetically determined shape and size and a tightly integrated developmental program. Plant tissue is shaded in green.}\label{fig:asexual-structures-fungi}
\end{figure}

\end{columns}
\end{frame}

\begin{frame}{}
\protect\hypertarget{section-3}{}
\begin{itemize}
\tightlist
\item
  \emph{Puccinia graminis} is a macrocyclic (produces one or more of
  spermatia, aeciospores and uredospores) heteroecious fungus that
  causes wheat stem rust disease.
\item
  The repeating stage (allows the disease to persist) in this fungus
  occurs on wheat and not the alternate host, barberry.
\end{itemize}

\begin{columns}[T, onlytextwidth]
\column{0.5\textwidth}

\begin{figure}
\includegraphics[width=0.8\linewidth]{../images/stem_rust_wheat_puccinia_graminis_f.sp.tritici} \caption{Stem rust of Wheat (\textit{Triticum aestivium}) caused by \textit{Puccinia graminis} f. sp. tritici, here with urediniospores and teliospores (First pic shows telia and the second a mix of telia and uredinia).}\label{fig:puccinia-graminis-wheat}
\end{figure}

\column{0.5\textwidth}

\begin{figure}
\includegraphics[width=0.85\linewidth]{../images/berberry_aecium} \caption{Wheat rust infection on barberry; Cross section of infected leaf, spermagonium on upper leaf, leafy bough of barberry, aecium with aeciospores on lower leaf surface. These spores are dikaryotic (they each have two haploid nuclei of different genetic make-up). These go on to infect wheat. The aeciospores are borne in chains and are held together by little rectangular cells called disjunctors.}\label{fig:berberry-aecium}
\end{figure}

\end{columns}
\end{frame}

\begin{frame}{}
\protect\hypertarget{section-4}{}
\begin{itemize}
\tightlist
\item
  Because there is no repeating stage in the life cycle of demicyclic
  fungi, removal of the primary or the alternate host will disrupt the
  disease cycle.
\item
  Cedar-apple rust disease, for example, can persist despite removal of
  one of the hosts since spores can be disseminated from long distances

  \begin{itemize}
  \tightlist
  \item
    severity of this disease can be managed by removal of basidiospore
    producing galls from junipers or the application of protective
    fungicides to junipers.
  \end{itemize}
\item
  Sulphur powder is known to stop spore germination. Fungicides such as
  Mancozeb and Triforine may help but may never eradicate the disease.
\end{itemize}
\end{frame}

\begin{frame}{UG99}
\protect\hypertarget{ug99}{}
\begin{columns}[T, onlytextwidth]
\column{0.6\textwidth}

It is a lineage of wheat stem rust ( \textit{Puccinia graminis} f. sp. \textit{tritici}), which is present in wheat fields in several countries in Africa and Middle east and is predicted to spread rapidly through these regions and possibly further afield, potentially causing a wheat production disaster that would affect food security worldwide. It can cause up to 100% crop losses and is virulent against many resistance genes which have previously protected against stem rust.

\column{0.4\textwidth}

\begin{figure}
\includegraphics[width=0.6\linewidth]{../images/B5_StemRust} \caption{Stem rust (race B5) on Wheat.}\label{fig:stem-rust}
\end{figure}

\end{columns}
\end{frame}

\begin{frame}{}
\protect\hypertarget{section-5}{}
\begin{columns}[T, onlytextwidth]
\column{0.4\textwidth}

\begin{figure}
\includegraphics[width=0.45\linewidth]{../images/ustilago_avenae} \caption{Smut head of Oat caused by \textit{Ustilago avenae}.}\label{fig:avenae-smut}
\end{figure}

\column{0.6\textwidth}

\begin{figure}
\includegraphics[width=0.95\linewidth]{../images/FdhD0Z3XoAIf1OB} \caption{Rust on White poplar \textit{Populus alba} caused by \textit{Melampsora} sp.. Uredinia are protruding on the upper face of some leaves and in cluster on ribs.}\label{fig:poplar-rust}
\end{figure}

\end{columns}
\end{frame}

\begin{frame}{}
\protect\hypertarget{section-6}{}
\begin{figure}\caption{Stem rust on UK breeding nurseries}\begin{columns}\column{0.5\textwidth}

\begin{center}\includegraphics[width=0.55\linewidth]{../images/FWGDjevXwAEBzeT} \end{center}

\column{0.5\textwidth}

\begin{center}\includegraphics[width=0.55\linewidth]{../images/FWGDkkkWYAA3yF4} \end{center}

\end{columns}\end{figure}

\footnotesize

\begin{itemize}
\tightlist
\item
  Wheat stripe rust is an air-borne and destructive disease caused by a
  heteroecious rust fungus \textit{Puccinia striiformis }f.~sp.
  \textit{tritici} (Pst). Studies have demonstrated that the rust
  pathogen accomplishes sexual reproduction on susceptible barberry
  under natural conditions in spring, whereas Pst infection on barberry
  is still in blank in other seasons.
\end{itemize}
\end{frame}

\begin{frame}{}
\protect\hypertarget{section-7}{}
\begin{figure}\caption{\textit{Puccinia recondita} on \textit{Secale cereale}} 
\begin{columns}\column{0.4\textwidth}\includegraphics[width=0.72\linewidth]{../images/FYCo7NmWQAkXvRB}\column{0.6\textwidth}\includegraphics[width=0.6\linewidth]{../images/FYCo7NwXgAYe9r1} \\ 
\includegraphics[width=0.6\linewidth]{../images/FYCo7NrXkAAOJZh}\end{columns}\end{figure}
\end{frame}

\begin{frame}{}
\protect\hypertarget{section-8}{}
\begin{figure}
\includegraphics[width=0.36\linewidth]{../images/FRruRoiX0AE_dF0} \caption{Rust on flowering stage \textit{Berberis vulgaris} caused by \textit{Puccinia graminis} at the beginning of the aecial stage.}\label{fig:berberis-rust}
\end{figure}

\begin{figure}
\includegraphics[width=0.38\linewidth]{../images/Fa1rQDGWAAAa8pu} \caption{Rust on a wild pear (\textit{Pyrus communis}) likely caused by \textit{Gymnosporangium sabinae}, here with a cluster of pycnia and pre-aecial tissues parasitized by \textit{Tuberculina} sp. (A hyperparasite) which is specific to Gymnosporangium.}\label{fig:pear-rust}
\end{figure}
\end{frame}

\begin{frame}{Fusarium}
\protect\hypertarget{fusarium}{}
\begin{itemize}
\tightlist
\item
  Fusariusm head blight/ear blight, foot rot, seedling blight Pathogen:
  \emph{Fusarium spp.} and \emph{Microdochium nivale}
\item
  Hosts: Wheat, barley, oats, rye triticale and grasses.
\item
  Symptoms

  \begin{itemize}
  \scriptsize
  \item Form a complex of diseases on seeds, seedlings and adult plants.
  \item \textit{Microdochium nivale} (formerly known as \textit{Fusarium nivale}) is seed-borne pathogen and causes seedling blight resulting in seedling death and thinning of plant stand.
  \item \textit{M}. spp (other than \textit{M. nivale}) cause a range of symptoms including brown lesions on stem bases, often restricted to outer leaf sheath.
  \item \textit{Fusarium lesions} often begin in the leaf sheath at the stem base where crown roots split the leaf sheath when emerging.
  \item This infection can spread up the leaf sheath causing long dark brown streaks at the stem base. The other symptom in cooler regions is brown staining of lower nodes.
  \item In older plants, fusarium infection can produce a true foot rot, where the stem base becomes brown and rotten, resulting in lodging and white heads.
  \item Symptoms are prevalent in very dry seasons as well.
  \item Ear blight causing fungus: \textit{F culmorum} and \textit{F graminearum} are common. Other are, \textit{F avenaceum}, \textit{F poae} and \textit{F langsethiae}.
  \item Infection frequently results in the whole or part of the ear becoming bleached.
  \item Symptoms seen when ears become infected during the early flowering stages, later infection may result in infection of grain but without obvious bleaching of the ears.
  \item Important due to its mycotoxin that gets accumulated in grains.
  \end{itemize}
\end{itemize}
\end{frame}

\begin{frame}{Fusarium: Life cycle}
\protect\hypertarget{fusarium-life-cycle}{}
\begin{itemize}
\tightlist
\item
  Most important source is seed but fungus survives on debris in soil
  also.
\item
  Spores are splashed in canopy causing ear blights and seed borne
  infection, in wet seasons, especially during flowering and grain
  formation.
\item
  Most fusarium species have competative saprophytic abilities which
  allow them to colonize debris and stubble in soil.
\item
  Importance:

  \begin{itemize}
  \tightlist
  \item
    When wet season coincides with flowering high levels of ear blight
    can occur.
  \item
    Due to seed borne nature of pathogen, seed treatment plays role in
    preventing seedling loss in wheat.
  \end{itemize}
\end{itemize}
\end{frame}

\begin{frame}{}
\protect\hypertarget{section-9}{}
\begin{figure}
\includegraphics[width=0.38\linewidth]{../images/plantain_podosphaera_plantaginis_powdery_mildew} \caption{Powdery mildew on buckhorn plantain (\textit{Plantago lanceolata}) caused by \textit{Podosphaera plantaginis}, here with cleistothecia containing a single ascus and eight ascospores.}\label{fig:plantain-powdery-mildew}
\end{figure}

\begin{figure}
\includegraphics[width=0.38\linewidth]{../images/blackberry_botrytis_rot} \caption{Botrytis fruit rot on blackberry (\textit{Rubus fruticosus}).}\label{fig:blackberry-botrytis-rot}
\end{figure}
\end{frame}

\begin{frame}{}
\protect\hypertarget{section-10}{}
\begin{columns}[T, onlytextwidth]
\column{0.5\textwidth}

\begin{figure}
\includegraphics[width=0.9\linewidth]{../images/suppression_rice_miR168} \caption{Suppression of rice miR168 improves yield, flowering time and immunity against rice blast using a target mimic approach.}\label{fig:rice-blast-tolerance}
\end{figure}

\column{0.5\textwidth}

\begin{figure}
\includegraphics[width=0.9\linewidth]{../images/biology_rice_blast_magnaporthe_infection} \caption{Cell and developmental biology of plant infection by rice blast fungus \textit{Magnaporthe oryzae}.}\label{fig:rice-blast-biology}
\end{figure}

\end{columns}

\footnotesize Note: For cell and developmental biology of infection by
blast fungus, refer to \citet{eseola2021investigating}.
\end{frame}

\begin{frame}{}
\protect\hypertarget{section-11}{}
\begin{columns}[T, onlytextwidth]
\column{0.45\textwidth}

\begin{figure}
\includegraphics[width=0.84\linewidth]{../images/panicle_bast_rice} \caption{Panicle blast of rice caused by \textit{Burkholderia glumae}}\label{fig:rice-panicle-blast}
\end{figure}

\column{0.55\textwidth}

\begin{figure}
\includegraphics[width=0.85\linewidth]{../images/citrus_peel_for_blast_control} \caption{Tangeretin, an antioxidant commonly found in citrus peels, is a powerful antifungal in the fight against rice blast disease.}\label{fig:citrus-peel-rescuse}
\end{figure}

\end{columns}

\footnotesize Refer to \citet{liang2021tangeretin} on use of antioxidant
against rice blast fungus.
\end{frame}

\begin{frame}{}
\protect\hypertarget{section-12}{}
\begin{figure}
\includegraphics[width=0.25\linewidth]{../images/rice_hoja_blanca_virus} \caption{Rice hoja blanca virus}\label{fig:rice-hoja-blanca-virus}
\end{figure}
\end{frame}

\hypertarget{insects}{%
\section{Insects}\label{insects}}

\begin{frame}{Major insects of Rice}
\protect\hypertarget{major-insects-of-rice}{}
\renewcommand{\arraystretch}{0.6}

\begin{table}

\caption{\label{tab:insects-damage}Major insects of rice and nature of their damange}
\centering
\fontsize{5}{7}\selectfont
\begin{tabular}[t]{>{\raggedright\arraybackslash}p{8em}>{\raggedright\arraybackslash}p{14em}>{\raggedright\arraybackslash}p{18em}}
\toprule
Damaged part & Insect & Scientific name\\
\midrule
 & Rice ear cutting caterpillar & \textit{Mythimna seperata}\\
\cmidrule{2-3}
 & Rice swarming caterpillar & \textit{Spodoptera mauritia}\\
\cmidrule{2-3}
 & Rice leaf folder & \textit{Cnaphalocrosis medinalis}\\
\cmidrule{2-3}
 & Rice caseworm & \textit{Nymphyla dpunctalis}\\
\cmidrule{2-3}
 & Rice grasshopper & \textit{Hyeroglyphus banian}\\
\cmidrule{2-3}
 & Rice hispa & \textit{Dickladispa armigera}\\
\cmidrule{2-3}
\multirow{-7}{8em}{\raggedright\arraybackslash Leaf} & Field cricket & \textit{Gryllus bimaculatus}\\
\cmidrule{1-3}
 & Yellow stem borer & \textit{Scripophaga incertulus}\\
\cmidrule{2-3}
 & Rice pink borer & \textit{Sesamia inferens}\\
\cmidrule{2-3}
 & Gall midge & \textit{Orselolia oryzae}\\
\cmidrule{2-3}
\multirow{-4}{8em}{\raggedright\arraybackslash Stem} & Striped stem borer & \textit{Chilo partellus}\\
\cmidrule{1-3}
 & Brown plant hopper & \textit{Nilaparvata lugens}\\
\cmidrule{2-3}
 & Rice thrips & \textit{Stenochaetothrips biformis}\\
\cmidrule{2-3}
\multirow{-3}{8em}{\raggedright\arraybackslash Tender shoots} & Green leaf hopper & \textit{Nephotettix virisens}\\
\cmidrule{1-3}
Root & Mole cricket & \textit{Gryllotalpa africana}\\
\cmidrule{1-3}
 & Rice earhead bug & \textit{Leptocorisa oratorious}\\
\cmidrule{2-3}
\multirow{-2}{8em}{\raggedright\arraybackslash Grain, flower} & Flower/pollen beetle & \textit{Chiloloba acuta}\\
\bottomrule
\end{tabular}
\end{table}
\end{frame}

\begin{frame}{Major Insects of Fruits and Vegetables}
\protect\hypertarget{major-insects-of-fruits-and-vegetables}{}
\renewcommand{\arraystretch}{0.6} 
\begin{table}

\caption{\label{tab:major-insects-fruits-vegetables}Insect pests of fruits and vegetables}
\centering
\fontsize{4}{6}\selectfont
\begin{tabular}[t]{>{\raggedright\arraybackslash}p{6em}>{\raggedright\arraybackslash}p{10em}>{\raggedright\arraybackslash}p{14em}>{\raggedright\arraybackslash}p{12em}>{\raggedright\arraybackslash}p{8em}>{\raggedright\arraybackslash}p{10em}}
\toprule
Crop group & Common name & Scientific name & Family & Control & Remark\\
\midrule
 & Wolly aphid & Eriosoma langierum & Hemiptera &  & \\
\cmidrule{2-6}
 & San jose scale & Quadraspidiotus perniciosus & Hemiptera &  & \\
\cmidrule{2-6}
 & Stem borer & Apriona cinera & Coleoptera &  & \\
\cmidrule{2-6}
 & Root borer & Dorysthenes hugeli & Coleoptera &  & \\
\cmidrule{2-6}
\multirow{-5}{6em}{\raggedright\arraybackslash Apple} & Codling moth & Cydia pomonella & Lepidoptera &  & \\
\cmidrule{1-6}
 & Rhizome weevil & Cosmopolites sordidus & Coleoptera &  & \\
\cmidrule{2-6}
 & Pseudostem weevil & Odoiporus longicollis & Coleoptera &  & \\
\cmidrule{2-6}
 & Skipper & Erionata thrax thrax & Lepidoptera &  & \\
\cmidrule{2-6}
 & Leaf and fruit scarring beetle & Nodostoma viridipennis & Coleoptera &  & \\
\cmidrule{2-6}
 & Aphid & Pentalonia nigronervosa & Hemiptera &  & \\
\cmidrule{2-6}
 & Lace-wing bug & Stephanitis typica & Hemiptera &  & \\
\cmidrule{2-6}
\multirow{-7}{6em}{\raggedright\arraybackslash Banana} & Fruit fly & Bactrocera musae & Hemiptera (Tephritidae) &  & \\
\cmidrule{1-6}
 & Tent caterpillar & Malacosoma indicum & Lepidoptera &  & Also affects apricot and walnut\\
\cmidrule{2-6}
\multirow{-2}{6em}{\raggedright\arraybackslash Peach} & Leaf curl aphid & Brachycaudia helichrysi & Hemiptera &  & \\
\bottomrule
\end{tabular}
\end{table}
\end{frame}

\begin{frame}{}
\protect\hypertarget{section-13}{}
\begin{table}

\caption{\label{tab:unnamed-chunk-1}Insect pests of fruits and vegetables (...continued)}
\centering
\fontsize{4}{6}\selectfont
\begin{tabular}[t]{>{\raggedright\arraybackslash}p{5em}>{\raggedright\arraybackslash}p{13em}>{\raggedright\arraybackslash}p{14em}>{\raggedright\arraybackslash}p{14em}>{\raggedright\arraybackslash}p{7em}>{\raggedright\arraybackslash}p{12em}}
\toprule
Crop group & Common name & Scientific name & Family & Control & Remark\\
\midrule
 & Citrus psylla & Diaphorina citri & Hemiptera &  & \\
\cmidrule{2-6}
 & Oriental fruit fly & Bactocera dorsalis & Diptera (Tephritidae) &  & \\
\cmidrule{2-6}
 & Stink bug & Rhynchocoris poseidon & Hemiptera &  & \\
\cmidrule{2-6}
 & Red scale & Aonideiella aurantii & Hemiptera &  & \\
\cmidrule{2-6}
 & Citrus aphid & Toxoptera citricidus & Hemiptera &  & \\
\cmidrule{2-6}
 & Orange stem borer & Stomatimum barbatum & Coleoptera &  & \\
\cmidrule{2-6}
\multirow{-7}{5em}{\raggedright\arraybackslash Citrus} & Citrus mealy bug & Planococcus citri & Hemiptera &  & \\
\cmidrule{1-6}
 & Hopper & Idioscopus nitidulus & Hemiptera (Cicadellidae) &  & \\
\cmidrule{2-6}
 & Fruit fly & Bactrocera dorsalis & Hemiptera (Tephritidae) &  & \\
\cmidrule{2-6}
 & Mealy bug & Drosicha mangiferae & Hemiptera &  & \\
\cmidrule{2-6}
 & Stem borer & Bactrocera rufomaculata & Coleoptera &  & \\
\cmidrule{2-6}
 & Leaf webber & Orthaga spp. & Lepidoptera &  & \\
\cmidrule{2-6}
 & Stone weevil & Sternochetus mangiferae & Coleoptera &  & \\
\cmidrule{2-6}
 & Gall psyllid & Apsylla cistella & Hemiptera &  & \\
\cmidrule{2-6}
\multirow{-8}{5em}{\raggedright\arraybackslash Mango} & Slug caterpillar, Mango leaf cutting weevil &  &  &  & \\
\bottomrule
\end{tabular}
\end{table}
\end{frame}

\begin{frame}{}
\protect\hypertarget{section-14}{}
\begin{table}

\caption{\label{tab:unnamed-chunk-2}Insect pests of fruits and vegetables (...continued)}
\centering
\fontsize{5}{7}\selectfont
\begin{tabular}[t]{>{\raggedright\arraybackslash}p{6em}>{\raggedright\arraybackslash}p{12em}>{\raggedright\arraybackslash}p{14em}>{\raggedright\arraybackslash}p{14em}>{\raggedright\arraybackslash}p{8em}>{\raggedright\arraybackslash}p{8em}}
\toprule
Crop group & Common name & Scientific name & Family & Control & Remark\\
\midrule
 & Rice weevil & Sitophus oryzae & Coleoptera (Curculionidae) &  & \\
\cmidrule{2-6}
 & Maize weevil & Sitophus zeamais & Coleoptera (Curculionidae) &  & \\
\cmidrule{2-6}
 & Angoumois grain moth & Sitotroga cerealella & Lepidoptera (Gelechiidae) &  & \\
\cmidrule{2-6}
 & Lesser grain borer & Rhizopertha dominica & Coleoptera (Bostrichidae) &  & \\
\cmidrule{2-6}
 & Rice moth & Corcyra cephalonica & Lepidoptera (Pyralidae) &  & \\
\cmidrule{2-6}
 & Khapra beetle & Trogoderma granarium & Coleoptera (Dermastidae) &  & \\
\cmidrule{2-6}
 & Pulse beetle & Callosobruchus chinensis & Coleoptera (Bruchidae) &  & \\
\cmidrule{2-6}
 & Cowpea beetle & Callosobruchus maculates & Coleoptera (Bruchidae) &  & \\
\cmidrule{2-6}
 & Rust red flour beetle & Tribolium castaneum & Coleoptera (Tenebrionidae) &  & \\
\cmidrule{2-6}
 & Confused flour beetle & Tribolium confusum & Coleoptera (Tenebrionidae) &  & \\
\cmidrule{2-6}
 & Warehouse moth & Ephestia cautella & Lepidoptera (Pyralidae) &  & \\
\cmidrule{2-6}
 & Indian meal moth & Plodia interpunctella & Lepidoptera (Phycitidae) &  & \\
\cmidrule{2-6}
 & Bean weevil & Acanthoscelides obtectus & Coleoptera (Bruchidae) &  & \\
\cmidrule{2-6}
\multirow{-14}{6em}{\raggedright\arraybackslash Storage pests} & Granary weevil & Sitophilus granaries & Coleoptera (Curculionidae) &  & \\
\bottomrule
\end{tabular}
\end{table}
\end{frame}

\begin{frame}{}
\protect\hypertarget{section-15}{}
\begin{table}

\caption{\label{tab:unnamed-chunk-3}Insect pests of fruits and vegetables (...continued)}
\centering
\fontsize{4}{6}\selectfont
\begin{tabular}[t]{>{\raggedright\arraybackslash}p{6em}>{\raggedright\arraybackslash}p{12em}>{\raggedright\arraybackslash}p{15em}>{\raggedright\arraybackslash}p{14em}>{\raggedright\arraybackslash}p{6em}>{\raggedright\arraybackslash}p{16em}}
\toprule
Crop group & Common name & Scientific name & Family & Control & Remark\\
\midrule
 & Shoot and fruit borer & Leucinodes orbonalis & Lepidoptera &  & \\
\cmidrule{2-6}
 & Leaf folder & Eublemma olivacea & Lepidoptera &  & \\
\cmidrule{2-6}
 & Leaf webber & Herpetogramma bipuncatalis & Lepidoptera &  & \\
\cmidrule{2-6}
 & Spotted beetle & Epilachna vigintioctopunctata & Coleoptera &  & \\
\cmidrule{2-6}
\multirow{-5}{6em}{\raggedright\arraybackslash Brinjal} & Aphid, Cotton jassid, Tobaccoo caterpillar, Soybean hairy caterpillar, White fly, Green semilooper, Grasshopper, Red ant, White grub &  &  &  & Minor insects\\
\cmidrule{1-6}
 & Potato tuber moth & Phthorimaea operculella & Lepidoptera &  & \\
\cmidrule{2-6}
 & Red ant & Dorylus orientalis & Hymenoptera &  & \\
\cmidrule{2-6}
\multirow{-3}{6em}{\raggedright\arraybackslash Potato} & Silver white fly & Bemisia tabaci & Homoptera &  & Variants of white flies of economic importance are: Aleurocanthus woglumi (citrus blackfly), which, in spite of its color, is a whitefly that attacks citrus. Aleyrodes proletella (cabbage whitefly), is a pest of various Brassica crops. Trialeurodes vaporariorum (greenhouse whitefly), a major pest of greenhouse fruit, vegetables, and ornamentals.\\
\bottomrule
\end{tabular}
\end{table}
\end{frame}

\begin{frame}{}
\protect\hypertarget{section-16}{}
\begin{table}

\caption{\label{tab:unnamed-chunk-4}Insect pests of fruits and vegetables (...continued)}
\centering
\fontsize{4}{6}\selectfont
\begin{tabular}[t]{>{\raggedright\arraybackslash}p{5em}>{\raggedright\arraybackslash}p{10em}>{\raggedright\arraybackslash}p{15em}>{\raggedright\arraybackslash}p{12em}>{\raggedright\arraybackslash}p{6em}>{\raggedright\arraybackslash}p{22em}}
\toprule
Crop group & Common name & Scientific name & Family & Control & Remark\\
\midrule
 & American bollworm/Fruit borer & Helicoverpa armigera & Noctuidae &  & Bolls how regular, circular bore holes; A single larvae can damage 30-40 bolls; Inundative release of egg parasitoid, Trichogramma spp., at 6.25 cc/ha at 15 days interval, 3 times from 45 DAS; Releasing of predator Chrysoperla carnea 100000/ha at 6th, 13th and 14th week after sowing; During bolling and maturation apply one of the following (1000 liter per hectare spray): Quinalphos 25EC 2.0 liter per hectare, Carbaryl 50 WP 2.5 kg per hectare, Cypermethrin 10 EC 600-800 ml per hectare.\\
\cmidrule{2-6}
 & Pink bollworm & Pectinophora gossypiella & Lepidoptera (Gelechiidae) &  & \\
\cmidrule{2-6}
 & Spotted bollworm & Earias vittella & Lepidoptera (Noctuidae) &  & \\
\cmidrule{2-6}
 & Tobaccoo cutworm & Spodoptera litura & Lepidoptera (Noctuidae) &  & \\
\cmidrule{2-6}
 & Cotton aphid & Aphis gossypii & Hemiptera (Aphididae) &  & \\
\cmidrule{2-6}
 & Thrips & Thrips tabaci & Thysanoptera &  & \\
\cmidrule{2-6}
 & White fly & Bemisia tabaci & Hemiptera (Aleyrodidae) &  & \\
\cmidrule{2-6}
\multirow{-8}{5em}{\raggedright\arraybackslash Cotton} & Red cotton bug & Dysdercus cingulatus & Hemiptera (Pyrrhocoridae) &  & \\
\bottomrule
\end{tabular}
\end{table}
\end{frame}

\begin{frame}{}
\protect\hypertarget{section-17}{}
\begin{table}

\caption{\label{tab:unnamed-chunk-5}Insect pests of fruits and vegetables (...continued)}
\centering
\fontsize{4}{6}\selectfont
\begin{tabular}[t]{>{\raggedright\arraybackslash}p{6em}>{\raggedright\arraybackslash}p{12em}>{\raggedright\arraybackslash}p{14em}>{\raggedright\arraybackslash}p{14em}>{\raggedright\arraybackslash}p{14em}>{\raggedright\arraybackslash}p{8em}}
\toprule
Crop group & Common name & Scientific name & Family & Control & Remark\\
\midrule
 & Cabbage butterfly & Pieris brassicae & Lepidoptera & Dichlorovos 76 EC, Nuvan 1 ml, Malation 50\% EC 2ml & \\
\cmidrule{2-6}
 & Diamond-back moth & Plutella xylostella & Lepidoptera & DBM lure, Azadirachtin 0.003\% EC, Beauveria bassiana, Emamectin benzoate, Cypermethrin 10\% EC 2ml & \\
\cmidrule{2-6}
 & Tobaccoo caterpillar & Spodoptera litura & Lepidoptera & Spodo-lure & \\
\cmidrule{2-6}
 & Mustard aphid & Lipaphis erysimi & Homoptera & Augmentation of natural predators Lady bird beetle (Coccinella septumpunctata) and Syrphid fly; Malathion 50 EC 1.5-2 ml per liter of water & \\
\cmidrule{2-6}
 & Mustard sawfly & Athalia lugens & Hymenoptera & Summer ploughing & \\
\cmidrule{2-6}
 & Cutworm & Agrotis ipsilon, A. segetum & Lepidoptera &  & \\
\cmidrule{2-6}
 & Flea beetle & Phyllotreta cruciferae & Coleoptera &  & \\
\cmidrule{2-6}
\multirow{-8}{6em}{\raggedright\arraybackslash Cruciferous vegetables} & Semi looper & Thysanoplusia orichalcea & Lepidoptera &  & \\
\bottomrule
\end{tabular}
\end{table}
\end{frame}

\begin{frame}{}
\protect\hypertarget{section-18}{}
\begin{table}

\caption{\label{tab:unnamed-chunk-6}Insect pests of fruits and vegetables (...continued)}
\centering
\fontsize{4}{6}\selectfont
\begin{tabular}[t]{>{\raggedright\arraybackslash}p{6em}>{\raggedright\arraybackslash}p{12em}>{\raggedright\arraybackslash}p{14em}>{\raggedright\arraybackslash}p{14em}>{\raggedright\arraybackslash}p{14em}>{\raggedright\arraybackslash}p{8em}}
\toprule
Crop group & Common name & Scientific name & Family & Control & Remark\\
\midrule
 & Re pumpkin beetle & Aulacophora foevicolis & Coleoptera &  & \\
\cmidrule{2-6}
 & Cucurbit stink bug & Cordius janus & Hemiptera &  & \\
\cmidrule{2-6}
 & Pumpkin fruit fly & Bactrocera cucurbitae & Diptera &  & \\
\cmidrule{2-6}
 & Spotted beetle & Epilachna vigintioctopunctata & Coleoptera &  & \\
\cmidrule{2-6}
\multirow{-5}{6em}{\raggedright\arraybackslash Cucurbit crops} & Cutworm, Semi-looper, Flea beetle, Aphid, White fly, Stem boring beetle, Banded blister beetle & Mylabris orientalis &  &  & Minior insects\\
\cmidrule{1-6}
 & Aphid & Aphis gossypii, Myzus persicae & Hemiptera &  & \\
\cmidrule{2-6}
 & Pea leaf miner & Liriomyza huidobrensis & Agromyzidae &  & \\
\cmidrule{2-6}
\multirow{-3}{6em}{\raggedright\arraybackslash Solanaceous crops} & Cutworm, Spotted beetle, White grub, Wireworm, Tobaccoo caterpillar, Flea beetle, Soybean, Hairy caterpillar &  &  &  & Minor insects\\
\bottomrule
\end{tabular}
\end{table}
\end{frame}

\begin{frame}{}
\protect\hypertarget{section-19}{}
\begin{table}

\caption{\label{tab:unnamed-chunk-7}Insect pests of fruits and vegetables (...continued)}
\centering
\fontsize{5}{7}\selectfont
\begin{tabular}[t]{>{\raggedright\arraybackslash}p{6em}>{\raggedright\arraybackslash}p{10em}>{\raggedright\arraybackslash}p{14em}>{\raggedright\arraybackslash}p{12em}>{\raggedright\arraybackslash}p{8em}>{\raggedright\arraybackslash}p{8em}}
\toprule
Crop group & Common name & Scientific name & Family & Control & Remark\\
\midrule
 & Tomato fruit borer & Helicoverpa armigera & Lepidoptera &  & \\
\cmidrule{2-6}
 & Jassid & Amarasca biguttula & Hemiptera &  & \\
\cmidrule{2-6}
\multirow{-3}{6em}{\raggedright\arraybackslash Tomato} & Tomato leaf miner & Tuta absoluta & Lepidoptera &  & \\
\cmidrule{1-6}
 & White stem borer & Xylotrechus quadripes & Coleoptera &  & \\
\cmidrule{2-6}
 & Berry borer & Hypothenemus hampei & Coleoptera &  & \\
\cmidrule{2-6}
 & Shoot-hole borer & Xylosandrus compactus & Coleoptera &  & \\
\cmidrule{2-6}
 & Stripped mealy bug & Ferrisia virgata & Hemiptera &  & \\
\cmidrule{2-6}
 & Helmet scale & Saissetia coffeae & Hemiptera &  & \\
\cmidrule{2-6}
\multirow{-6}{6em}{\raggedright\arraybackslash Coffee} & Gren bug & Coccus viridis & Hemiptera &  & \\
\cmidrule{1-6}
 & Mosquito/Plant bug & Belopeltis theivora & Hemiptera &  & \\
\cmidrule{2-6}
 & Red spider mite & Oligonychus coffeae & Acarina &  & \\
\cmidrule{2-6}
 & Scarlet mite & Brevipalpus phoenicis & Acarina &  & \\
\cmidrule{2-6}
\multirow{-4}{6em}{\raggedright\arraybackslash Tea} & Purple mite & Calacarus carinatus & Acarina &  & \\
\bottomrule
\end{tabular}
\end{table}
\end{frame}

\hypertarget{bibliography}{%
\section{Bibliography}\label{bibliography}}

\begin{frame}{References}
\protect\hypertarget{references}{}
\end{frame}

          \begin{frame}[allowframebreaks]{}
    \bibliographytrue
    \bibliography{./../bibliographies.bib}
    \end{frame}
  


\end{document}
